Primeira Questão
Para calcularmos o vetor velocidade $$\vec{v}$$, basta derivarmos o vetor posição $\vec{r}$ em relação ao tempo: (Substituinto $\theta(t)$ por $\omega t$)
\begin{equation} \vec{r} = \left\langle R\cos\left(\omega t\right), R\sin\left(\omega t\right), ct \right\rangle \end{equation}

Aplicando a operação de derivada no vetor $\vec{r}$:
\begin{equation} \vec{v} = \dot{\vec{r}} = \left\langle -R \omega \sin\left(\omega t \right), R \omega \cos\left(\omega t \right), c \right\rangle \end{equation}

Assumindo que $\left\langle \hat{i}, \hat{j}, \hat{k} \right\rangle$ representam a base canônica (ortonormal), é possível calcular o módulo do vetor velocidade $\vec{v}$ com o produto escalar.
\begin{equation} |\vec{v}| = \sqrt{\vec{v} \cdot \vec{v}} = \sqrt{\dot{x}(t)^{2} + \dot{y}(t)^{2} + \dot{z}(t)^{2}} = \sqrt{\left( -R \omega \sin\left(\omega t\right) \right)^{2} + \left( R \omega \cos\left(\omega t\right) \right)^{2} + c^{2}} = \sqrt{R^{2} \omega^{2} + c^{2}} \end{equation}

Análise Dimensional


\begin{equation} \left[\vec{r}\right] = L \end{equation}

\begin{equation} \left[\frac{d \vec{r}}{dt}\right] = \frac{L}{T} = \left[\vec{v}\right] \end{equation} 

\begin{equation} \left[\sqrt{R^{2} \omega^{2} + c^{2}}\right] = \sqrt{ \left[R^{2} \omega^{2}\right]} = \sqrt{[c^{2}]} = \frac{L}{T} \end{equation}


Segunda Questão
O vetor aceleração pode ser calculado da mesma forma que foi feita na questão anterior, já que o vetor aceleração é a taxa de variação da velocidade e, portanto, pode ser calculado como a derivada da velocidade:
\begin{equation} \vec{v} = \left\langle -R \omega \sin\left(\omega t\right), R \omega \cos\left(\omega t \right), c \right\rangle \end{equation}
\begin{equation} \vec{a} = \dot{\vec{v}} = \left\langle -R \omega^{2} \cos\left(\omega t\right), -R \omega^{2} \sin\left(\omega t\right), 0 \right\rangle \end{equation}

O processo para calcular o módulo é o mesmo realizado anteriormente:
\begin{equation} |\vec{a}| = \sqrt{\vec{a} \cdot \vec{a}} = \sqrt{\left( -R \omega^{2} \cos\left(\omega t\right) \right)^{2} + \left( -R \omega^{2} \sin\left(\omega t\right) \right)^{2} + 0^{2}} = \sqrt{R^{2} \omega^{4}} = \omega^{2} R \end{equation}

Análise Dimensional
\begin{equation} [\vec{v}] = \frac{L}{T} \end{equation}
\begin{equation} \left[\frac{d \vec{v}}{dt}\right] = \frac{L}{T^{2}} = [\vec{a}] \end{equation}
\begin{equation} \left[\omega^{2} R\right] = \frac{L}{T^{2}} \end{equation}
Terceira Questão
O produto escalar entre $\vec{v}$ e $\vec{a}$ pode ser dado por:
\begin{equation} \vec{v} \cdot \vec{a} = \dot{x}(t) \cdot \ddot{x}(t) + \dot{y}(t) \cdot \ddot{y}(t) + \dot{z}(t) \cdot \ddot{z}(t) \end{equation}

Substituindo os valores de $\vec{v}$ e $\vec{a}$ vistos nas questões anteriores.
\begin{equation} \vec{v} \cdot \vec{a} = \left( -R \omega \sin\left(\omega t\right) \right) \cdot \left( -R \omega^{2} \cos\left(\omega t\right) \right) + \left( R \omega \cos\left(\omega t\right) \right) \cdot \left( -R \omega^{2} \sin\left(\omega t\right) \right) + c \cdot 0 = R^{2} \omega^{3} \sin\left(\omega t\right) \cos\left(\omega t\right) - R^{2} \omega^{3} \sin\left(\omega t\right) \cos\left(\omega t\right) \end{equation}

Como os dois termos são iguais e de sinais opostos: $\vec{v} \cdot \vec{a} = 0$. A explicação geométrica para isso é que os vetores velocidade e aceleração são perpendiculares entre si (já que ambos não são nulos).
\begin{equation} \vec{v} \cdot \vec{a} = 0 \end{equation}

Já o comprimento de uma dada trajetória $(S)$ entre os instantes $a$ e $b$ pode ser calculado como:
\begin{equation} S = \int_{a}^{b} | \vec{v}(t)| dt \end{equation}

Com $a = ti = 0$ e $b = tf = \frac{2 \pi}{\omega}$
\begin{equation} S = \int_{0}^{\frac{2 \pi}{\omega}} \sqrt{R^{2} \omega^{2} + c^{2}} dt \end{equation}

Como $\sqrt{R^{2} \omega^{2} + c^{2}}$ representa uma constante, pode ser colocada para fora da integral.
\begin{equation} S = \sqrt{R^{2} \omega^{2} + c^{2}} \int_{0}^{\frac{2 \pi}{\omega}} dt = \sqrt{R^{2} \omega^{2} + c^{2}} \frac{2 \pi}{\omega} \end{equation}

Para cálculo da distância $(L)$, podemos calcular a equação $\vec{r}(t)$ nos instantes $0$ e $\frac{2 \pi}{\omega}$ e aplicar a equação da distância entre dois pontos.\begin{equation} \vec{r}(0) = \left\langle R\cos\left(\omega \cdot 0\right), R\sin\left(\omega \cdot 0\right), c \cdot 0 \right\rangle = \left\langle R, 0, 0 \right\rangle \end{equation}
\begin{equation} \vec{r}\left(\frac{2 \pi}{\omega}\right) = \left\langle R\cos\left(\omega \frac{2 \pi}{\omega}\right), R\sin\left(\omega \frac{2 \pi}{\omega}\right), c \cdot \frac{2 \pi}{\omega} \right\rangle = \left\langle R, 0, \frac{2 \pi c}{\omega} \right\rangle \end{equation}

Aplicando a distância entre dois pontos (módulo da diferença entre estes dois vetores posição):
\begin{equation} \sqrt{ \left(x_a - x_b \right)^{2} + \left(y_a - y_b \right)^{2} + \left(z_a - z_b \right)^{2}} \end{equation}
\begin{equation} L = \sqrt{\left( R - R \right)^{2} + \left( 0 - 0 \right)^{2} + \left( 0 - \frac{2 \pi c}{\omega} \right)^{2}} = \frac{2 \pi c}{\omega} \end{equation}

É possível perceber baseado no comprimento da trajetória $S$ e na distância $L$, que ao "completar uma volta", a distância está no eixo $z$, pois em $x(t)$ e $y(t)$ há uma equação paramétrica da circunferência, logo, é esperado que $S > L$, pois o comprimento varia nos três eixos e percorre toda a circuferência, considerando o caminho total percorrido, enquanto a distância está limitada à variação do eixo $z$.
\begin{equation} \sqrt{R^{2} \omega^{2} + c^{2}} \frac{2 \pi}{\omega} > \frac{2 \pi c}{\omega} \end{equation}

Análise Dimensional
\begin{equation} [S] = \left[ |\vec{v}| dt \right] = \frac{L}{T} T = L \end{equation}
\begin{equation} \left[\sqrt{R^{2} \omega^{2} + c^{2}} \frac{2 \pi}{\omega}\right] = \frac{L}{T} \left[\frac{2 \pi}{\omega}\right] = L \end{equation}
\begin{equation} [L] = L \end{equation}
\begin{equation} \left[\frac{2 \pi c}{\omega}\right] = \frac{LT^{-1}}{T^{-1}} = L \end{equation}
Quarta Questão
Aplicando a operação de produto vetorial entre os vetores $\vec{v}$ e $\vec{a}$, temos o produto vetorial $\vec{v} \times \vec{a}$, utilizado para calcular a curvatura $\kappa$:
\begin{bmatrix} i & j & k \\ -R \omega \sin(\omega t) & R \omega \cos(\omega t) & c \\ -R \omega^2 \cos(\omega t) & -R \omega^2 \sin(\omega t) & 0 \end{bmatrix}
\begin{equation} \left\langle 0 - \left(-R \omega^{2} \sin(\omega t)\right) \cdot c, -R \omega^{2} \cos(\omega t) \cdot c, R^{2} \omega^{3} \sin^{2}(\omega t) - \left(-R^{2} \omega^{3} \cos^{2}(\omega t)\right) \right\rangle = \left\langle R \omega^{2}  \sin(\omega t) \cdot c, -R \omega^{2} \cos(\omega t) \cdot c, R^{2} \omega^{3} \right\rangle \end{equation}

Assim, aplicando novamente a equação do módulo:
\begin{equation} | \vec{v} \times \vec{a} | = \sqrt{\left(R \omega^{2} \sin \left(\omega t \right) \cdot c\right)^{2} + \left(-R \omega^{2} \cos \left(\omega t \right) \cdot c\right)^{2} + \left(R^{2} \omega^{3} \right)^{2}} = \sqrt{R^{2} \omega^{4} c^{2} + R^{4} \omega^{6}} = R \omega^{2} \sqrt{R^{2} \omega^{2} + c^{2}} \end{equation}

Aplicando na equação da curvatura $\kappa$
\begin{equation} \kappa = \frac{|\vec{v} \times \vec{a}|}{v^{3}} = \frac{R \omega^{2} \sqrt{R^{2} \omega^{2} + c^{2}}}{\left(\sqrt{R^{2} \omega^{2} + c^{2}}\right)^{3}} = \frac{R \omega^{2}}{R^{2} \omega^{2} + c^{2}} \end{equation}

Análise Dimensional
\begin{equation} [\kappa] = \left[\frac{|\vec{v} \times \vec{a}|}{v^{3}}\right] = \frac{\left[\frac{L}{T} \times \frac{L}{T^{2}}\right]}{\left(\frac{L}{T}\right)^{3}} = \frac{\frac{L^2}{T^3}}{\frac{L^3}{T^3}} = \frac{1}{L} \end{equation}

É possível perceber que o resultado da curvatura também pode ser escrito como $\frac{a}{v^{2}}$
\begin{equation} [\kappa] = \left[\frac{a}{v^{2}}\right] = \frac{\frac{L}{T^{2}}}{\frac{L^{2}}{T^{2}}} = \frac{1}{L} \end{equation}