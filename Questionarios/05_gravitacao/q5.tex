Introdução

O problema consiste em calcular o período de translação de um planeta de massa desprezível $$m$$ em relação a uma estrela de $$M = 8.7 \cdot 10^{30}kg$$, sabendo que os corpos se encontram a uma distância de $r = 5,4 \cdot 10^{11}m$. Logo, é possível assumir uma órbita circular de raio r com força centrípeta dada por: \begin{equation} \vec{F_c} = \frac{mv^{2}}{r} \hat{r} \end{equation} E seja a força gravitacional entre os corpos dada por: \begin{equation} \vec{F_g} = \frac{GMm}{r^{2}} \hat{r} \end{equation} Pela segunda lei de Newton, a força centrípeta deve ser igual à força gravitacional: \begin{equation} \vec{F_c} = \vec{F_g} \end{equation}
Velocidade e Velocidade Angular

Igualando as forças é possível obter a equação do módulo da velocidade: \begin{equation} v = \sqrt{\frac{MG}{r}} \end{equation} A partir disso é possível usar a relação $v = \omega r$ para descobrir a velocidade angular do movimento de translação: \begin{equation} \omega = \frac{\sqrt{\frac{MG}{r}}}{r} = \sqrt{\frac{MG}{r^{3}}} \end{equation}
Período
Para o cálculo do período, basta utilizar a equação $T = 2 \pi / \omega$ e substituir o valor da velocidade angular encontrado anteriormente: \begin{equation} T = 2 \pi \sqrt{\frac{r^{3}}{MG}} \end{equation} Utilizando os valores fornecidos: \begin{equation} T = 2 \pi \sqrt{\frac{(5.4 \cdot 10^{11})^{3}}{8.7 \cdot 10^{30} \cdot 6,673 \cdot 10^{-11}}} = 10.348 \cdot 10^{7} \end{equation} Portanto, o tempo necessário para que o planeta complete uma volta em torno da estrela é de $10.348 \cdot 10^{7}$.
Códigos
Python

import math

# Constantes dadas no problema
r = 5.4e11 # distância em metros
M = 8.7e30 # massa da estrela em kg
G = 6.673e-11 # constante gravitacional em m^3 kg^-1 s^-2

# Cálculo do período de translação
T = 2 * math.pi * math.sqrt(r**3 / (G * M))
print(round(T / 1e7, 3))

Maple

r := 0.54e12
M := 0.87e31
G := 0.6673e-10
T := 2*Pi*sqrt(r^3/(G*M))
evalf(T/0.1e8, 5) 